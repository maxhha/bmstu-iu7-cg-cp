\chapter{Технологический раздел}

В данном разделе будут определены средства программной реализации, описан процесс сборки, пользовательский интерфейс и приведено функциональное тестирование.

\section{Выбор средства программной реализации}

Для реализации был выбран фреймворк Qt\cite{qt} на С++, так как он содержит базовые компоненты пользовательского интерфейса, что позволит больше сконцентрироваться на программировании основного алгоритма. Также для Qt существует интегрированная среда разработки QtCreator\cite{qtcreator}, упрощающая создание пользовательского интерфейса и отладку приложения.

Были использованы следующие библиотеки:
\begin{itemize}
    \item \textbf{fmt}\cite{libfmt} - библиотека форматирования текста. Используется в генерации сообщений ошибок;
    \item \textbf{Eigen}\cite{libeigen} - библиотека для линейной алгебры. Используется только метод Якоби для решения системы линейных уравнений.
\end{itemize}

\section{Процесс сборки приложения}

Для сборки приложения из исходников понадобится Qt4.x\cite{qt4x} и QtCreator. В QtCreator нужно открыть папку проекта, выбрать сборку Release, собрать проект.

\section{Пользовательский интерфейс}

Пользовательский интерфейс состоит из области отображения полигональной модели и меню, разделенного по вкладкам. Каждая вкладка отражает один из этапов: загрузка томографии, преобразование в полигональную модель, настройка освещения и отображения модели, трансформация модели.

\imgs{ui_file}{th}{1}{Вкладка "Файл". На ней можно загрузить файл томографии, просмотреть информацию о загруженном файле, отредактировать размеры вокселя}

\imgs{ui_polygonize}{th}{1}{Вкладка "Полигонизация". На ней представлены параметры алгоритма полигонизации, кнопки сохранения и удаления созданной модели}

\clearpage

На рисунке \ref{img:ui_polygonize} видны дополнительные параметр \textit{Усреднение} и флаг \textit{Закрыть отверстия на гранях}.

В поле \textit{Усреднение} указывается расстояние до соседних вокселей, среди которых вычисляется среднее значение. Это позволяет получать более сглаженные модели на зашумленных данных томографии. 

Установленный флаг \textit{Закрыть отверстия на гранях} добавляет по одному пустому вокселю ко всем граням томографии, для закрытия отверстий, образовавшихся на гранях. Они появляются, потому что объект не поместился целиком в область томографии.

\imgs{ui_show}{th}{1}{Вкладка "Отображение". В ней можно настроить освещение и тени}

\clearpage

\imgs{ui_transform}{ht}{0.7}{Вкладка "Вид". В ней можно применить простые трансформации к модели для просмотра результата}

\section{Форматы входных и выходных данных}

Для хранения томографии был выбран формат файлов проекта OpenQVis\cite{file_oqvis} из-за своей простоты чтения и записи. Томография хранится в двух файлах: заголовочном (\texttt{.dat}) и файлом вокселей(\texttt{.raw}).

Для хранения полигональной модели был выбран формат файла STL(бинарная версия)\cite{file_stl}, так как является простым и широко поддерживаемым форматом.

\clearpage

\section{Пример работы приложения}

\imgw{work_screenshot}{th}{0.9\textwidth}{Пример результата работы приложения. Полигонизация томографии плюшевого мишки}

\section{Тестирование программного обеспечения}

Функциональное тестирование проведено с помощью известных геометрических объектов. Для этого программно были созданы их томографии. Размер томографии - (128, 128, 128). В таблице \ref{func_tests} представлены функции, описывающие объект, и результат полигонизации. Причем, функция $s$ - функция размывания границы, принимает знаковое расстояние до поверхности и коэффициент размытия,  имеет вид:

\begin{equation}
    s(v, w) = \begin{cases}
       1 &\quad\text{если}\ v < -w \\
       \frac{w - v}{2 w} &\quad\text{если}\  -w \le v < w \\
       0 &\quad\text{если}\ v \ge w
     \end{cases}
\end{equation}

\begin{table}
    \centering
    \caption{Функциональные тесты}
    \label{func_tests}
    \begin{tabularx}{\textwidth}{|X|X|X|}
        \hline
        Название & Функция $f$ & Результат \\
        \hline
        Шар с радиусом 40 & $s(\sqrt{x^2 + y^2 + z^2} - 40, 5)$ & \vspace{5mm} \includegraphics[]{./inc/img/func_test_ball} \\
        \hline
        Куб со стороной 80 & $s(\max(|x|, |y|, |z|) - \newline - 40, 0)$ & \vspace{5mm} \includegraphics[scale=0.8]{./inc/img/func_test_cube} \\
        \hline
         Цилиндр с радиусом основания 40 выстой 120 & $s(\sqrt{x^2 + y^2} - 40, 5) \cdot \newline \cdot s(|z| - 60, 0) $ & \vspace{5mm} \includegraphics[scale=0.8]{./inc/img/func_test_cylinder} \\
        \hline
        Куб со стороной 80, в котором вырезали шар радиусом 50 & $s(\max(|x|, |y|, |z|) - \newline - 40, 0) \cdot (1 - s(\newline\sqrt{x^2 + y^2 + z^2} - 50, 5))$ & \vspace{5mm} \includegraphics[scale=0.8]{./inc/img/func_test_cube_sub_ball} \\
        \hline
    \end{tabularx}
\end{table}

\section*{Вывод}
\addcontentsline{toc}{section}{Вывод}

В данном разделе были определены средства программной реализации, описан процесс сборки приложения, пользовательский интерфейс и приведено функциональное тестирование.