\chapter{Аналитический раздел}

В данном разделе будут рассмотрены алгоритмы компьютерной графики построения полигональных моделей из послойных снимков и выбраны наиболее подходящие алгоритмы.

\section{Формализация объектов}

\textbf{Томография} - получение послойного изображения внутренней структуры объекта (см. рис. \ref{img:tomography})

\imgs{tomography}{th}{0.2}{Томограммы (S1, S2) группы трёхмерных объектов и их проекция (P)}

\textbf{Результат томографии} - регулярная сетка вокселей, в которой каждому вокселю соответствует усредненное значение (температура, плотность материала) в данной точке трехмерного объекта. Данные результата томографии идеально подходят для хранения в трехмерном массиве. 

\textbf{Изоповерхность} - поверхность, представляющая точки с постоянным значением (например, плотности, давления, температуры, или скорости) в некоторой части пространства. Математически определяется как поверхность уровня:

\begin{equation}
    L_c(f) = \{ (x,y,z) | f(x,y,z) = c \}
    \label{eq:isoline}
\end{equation}

\textbf{Полигональная сетка} - совокупность вершин, ребер и граней, которые определяют форму многогранного объекта. 

\section{Анализ алгоритмов полигонизации}

Алгоритмы полигонизации разделяют на primal (прямые) и dual (двойственные). Dual алгоритмы в большинстве являются более поздними и способны более точно передать грани поверхности. Основное отличие их в том, как они строят полигоны из регулярной сетки значений функции. Primal ставят вершины на ребрах сетки, а dual определяют позицию вершины внутри куба сетки. Если представить узлы регулярной сетки вершинами графа, то можно сказать, что primal строят полигоны используя прямой граф, а dual двойственный к нему. На рисунке \ref{img:rpz_drawings-primal_vs_dual} показана разница между алгоритмами.

\imgw{rpz_drawings-primal_vs_dual}{th}{0.8\textwidth}{Красная линия -- аппроксимация изолинии отрезками. В primal алгоритмах вершины отрезков расположены на ребрах сетки, а в dual алгоритмах -- внутри ячеек сетки}

\subsection{Алгоритм Marching cubes ("шагающих кубов")}
Алгоритм шагающих кубов \cite{mc} рассматривает каждый куб в регулярной сетке, анализирует значения в вершинах куба, и определяет необходимые для представления части изоповерхности полигоны, проходящей через данный куб. Так как алгоритм выбирает полигоны, исходя только из положения вершин куба относительно изоповерхности, всего получается 256 ($2^8$) возможных конфигураций полигонов, которые можно вычислить заранее и сохранить в массиве.

\imgw{mc_cube_variants}{th}{0.8\textwidth}{256 возможных конфигураций куба могут быть сведены к 15 благодаря симметрии}

Этот алгоритм был опубликован первее остальных и является основным, наиболее часто используемым. Однако, он создает большое количество полигонов - от 1 на каждый пересекающий изоповерхность куб. Это значительно увеличивает нагрузку на отрисовку и какое-либо редактирование. Углы получаются сглаженными. Существуют случаи с неопределенностью (см. рис. \ref{img:rpz_drawing-mc_error_case}), который необходимо разрешать дополнительной проверкой соседних вершин.

\imgw{rpz_drawings-mc_error_case}{th}{0.8\textwidth}{Случай неопределенности в двумерном варианте}

\subsection{Алгоритм Marching Tetrahedra}

Алгоритм Marching Tetrahedra\cite{marching_tetrahedra} решает случай неопределенности из предыдущего алгоритма путем разбиения куб на тетраэдры (см. рис. \ref{img:marching_tetrahedrons}). Тетраэдр имеет 16 ($2^4$) возможных конфигураций полигонов.

Изначально этот алгоритм был придуман, как открытый аналог Marching Cubes, так как второй был запатентован. Однако, в наше время патент на Marching Cubes закончился.

\imgw{marching_tetrahedrons}{th}{0.4\textwidth}{Куб, составленный из 6 тетраэдров}

\subsection{Алгоритм Dual Contouring}

Алгоритм Dual Contouring\cite{dualcontouring} решает проблемы Marching Cubes c помощью градиента функции. Эта информация позволит создавать острые границы, разрешать неопределенные случаи.
Dual Contouring помещает в каждую ячейку по одной вершине, а затем соединяет точки, создавая полигональную сетку. Точки соединяются вдоль каждого ребра, имеющего смену знака, как и в Мarching Сubes. На рисунке \ref{img:rpz_drawings-dual_contouring} представлен пример работы алгоритма. 

\imgw{rpz_drawings-dual_contouring}{th}{0.3\textwidth}{Пример работы алгоритма Dual Contouring}

На рисунке \ref{img:rpz_drawings-dual_contouring} в вершинах сетки показаны направления градиентов функции. Вершины внутри ячеек расставляются таким образом, что построенная изолиния должна быть перпендикулярна этим градиентам

Dual Contouring создаёт более естественные формы, чем Marching Cubes и не требует создания таблицы конфигураций.

\subsection{Алгоритм Dual Marching Cubes}

Алгоритм Dual Marching Cubes\cite{dmc} является некоторой комбинацией Marching cubes и Dual Contouring. Его главной особенностью считается построение полигональной сетки без избыточного деления на полигоны. Это достигается благодаря построению восьмеричного дерева, в котором куб делится на 8 меньших кубов. В результате разбиения в каждом кубе существует такая точка, касательная плоскость к которой наилучшим образом описывает участок изоповерхность внутри этого куба. С помощью дерева каждая вершина соединяется ребром с соседними к ней. Далее, рассматривая полученный граф, как регулярную сетку, применяют алгоритм Marching Squares. На рисунке \ref{img:dmc_compare} приведено сравнение алгоритмов полигонизации\cite{dmc} на примере комнаты , построенной с использованием конструктивной блочной геометрией.

\imgw{dmc_compare}{th}{0.6\textwidth}{Слева вверху - комната-оригинал, построенная с использованием конструктивной блочной геометрией. Слева снизу - результат работы Marching Cubes (67 тыс. полигонов). Справа снизу - Dual Contouring (17 тыс. полигонов). Сверху справа - Dual Marching Cubes (440 полигонов)}

В результате работы алгоритма получается полигональная сетка с меньшим количеством полигонов и большей детализацией.

\section*{Вывод}
\addcontentsline{toc}{section}{Вывод}
В данном разделе были изучены и проанализированы алгоритмы полигонизации. В качестве решения были выбраны алгоритмы Dual Marching Cubes, потому что создает наиболее оптимизированную полигональную сетку, не теряющую деталей.
