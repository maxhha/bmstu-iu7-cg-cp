\chapter*{ВВЕДЕНИЕ}
\addcontentsline{toc}{chapter}{ВВЕДЕНИЕ}

С продвижением прогресса появляется необходимость в инструментах, находящихся на пересечении нескольких наук. Одним из таких пересечений является медицина и компьютерная графика. С помощью магнитно-резонансной томографии и компьютерной томографии собирают информацию о внутренней структуре органов и тканей. Затем эти данные должны быть корректно отображены на экране компьютера, чтобы медицинский специалист смог поставить диагноз.

В то же время, сейчас активно развиваются и применяются технологии трехмерной печати. В медицине они используются для создания протезов и имплантов. Индивидуально напечатанные протезы значительно увеличивают качество жизни. Но для их создания необходима информация о внутренней структуре заменяемого органа. Таким образом, возникает потребность в программном обеспечении, позволяющем по трехмерным снимкам получать файлы для трехмерной печати.

Целью моей работы реализация программы с пользовательским интерфейсом для создания полигональной модели по результатом томографии.

Для достижения поставленной цели необходимо выполнить следующие задачи: 

\begin{itemize}
    \item изучить и проанализировать алгоритмы компьютерной графики построения полигональных моделей из послойных снимков, выбрать наиболее подходящий алгоритм;
    \item подробно изучить выбранный алгоритм для применения к поставленной задаче;
    \item привести схему рассматриваемого алгоритма;
    \item описать используемые структуры данных;
    % \item оценить объем памяти, необходимый для хранения данных;
    \item описать структуру разрабатываемого программного обеспечения;
    \item определить средства программной реализации;
    \item описать процесс сборки приложения;
    \item протестировать разработанное программное обеспечение;
\end{itemize}

